\documentclass[10pt, letterpaper]{article}

% Packages:
\usepackage[
    ignoreheadfoot, % set margins without considering header and footer
    top=2 cm, % seperation between body and page edge from the top
    bottom=2 cm, % seperation between body and page edge from the bottom
    left=2 cm, % seperation between body and page edge from the left
    right=2 cm, % seperation between body and page edge from the right
    footskip=1.0 cm, % seperation between body and footer
    % showframe % for debugging 
]{geometry} % for adjusting page geometry
\usepackage{titlesec} % for customizing section titles
\usepackage{tabularx} % for making tables with fixed width columns
\usepackage{array} % tabularx requires this
\usepackage[dvipsnames]{xcolor} % for coloring text
\definecolor{primaryColor}{RGB}{0, 0, 0} % define primary color
\usepackage{enumitem} % for customizing lists
\usepackage{fontawesome5} % for using icons
\usepackage{amsmath} % for math
\usepackage[
    pdftitle={John Doe's CV},
    pdfauthor={John Doe},
    pdfcreator={LaTeX with RenderCV},
    colorlinks=true,
    urlcolor=primaryColor
]{hyperref} % for links, metadata and bookmarks
\usepackage[pscoord]{eso-pic} % for floating text on the page
\usepackage{calc} % for calculating lengths
\usepackage{bookmark} % for bookmarks
\usepackage{lastpage} % for getting the total number of pages
\usepackage{changepage} % for one column entries (adjustwidth environment)
\usepackage{paracol} % for two and three column entries
\usepackage{ifthen} % for conditional statements
\usepackage{needspace} % for avoiding page brake right after the section title
\usepackage{iftex} % check if engine is pdflatex, xetex or luatex

% Ensure that generate pdf is machine readable/ATS parsable:
\ifPDFTeX
    \input{glyphtounicode}
    \pdfgentounicode=1
    \usepackage[T1]{fontenc}
    \usepackage[utf8]{inputenc}
    \usepackage{lmodern}
\fi

\usepackage{charter}

% Some settings:
\raggedright
\AtBeginEnvironment{adjustwidth}{\partopsep0pt} % remove space before adjustwidth environment
\pagestyle{empty} % no header or footer
\setcounter{secnumdepth}{0} % no section numbering
\setlength{\parindent}{0pt} % no indentation
\setlength{\topskip}{0pt} % no top skip
\setlength{\columnsep}{0.15cm} % set column seperation
\pagenumbering{gobble} % no page numbering

\titleformat{\section}{\needspace{4\baselineskip}\bfseries\large}{}{0pt}{}[\vspace{1pt}\titlerule]

\titlespacing{\section}{
    % left space:
    -1pt
}{
    % top space:
    0.3 cm
}{
    % bottom space:
    0.2 cm
} % section title spacing

\renewcommand\labelitemi{$\vcenter{\hbox{\small$\bullet$}}$} % custom bullet points
\newenvironment{highlights}{
    \begin{itemize}[
        topsep=0.10 cm,
        parsep=0.10 cm,
        partopsep=0pt,
        itemsep=0pt,
        leftmargin=0 cm + 10pt
    ]
}{
    \end{itemize}
} % new environment for highlights


\newenvironment{highlightsforbulletentries}{
    \begin{itemize}[
        topsep=0.10 cm,
        parsep=0.10 cm,
        partopsep=0pt,
        itemsep=0pt,
        leftmargin=10pt
    ]
}{
    \end{itemize}
} % new environment for highlights for bullet entries

\newenvironment{onecolentry}{
    \begin{adjustwidth}{
        0 cm + 0.00001 cm
    }{
        0 cm + 0.00001 cm
    }
}{
    \end{adjustwidth}
} % new environment for one column entries

\newenvironment{twocolentry}[2][]{
    \onecolentry
    \def\secondColumn{#2}
    \setcolumnwidth{\fill, 4.5 cm}
    \begin{paracol}{2}
}{
    \switchcolumn \raggedleft \secondColumn
    \end{paracol}
    \endonecolentry
} % new environment for two column entries

\newenvironment{threecolentry}[3][]{
    \onecolentry
    \def\thirdColumn{#3}
    \setcolumnwidth{, \fill, 4.5 cm}
    \begin{paracol}{3}
    {\raggedright #2} \switchcolumn
}{
    \switchcolumn \raggedleft \thirdColumn
    \end{paracol}
    \endonecolentry
} % new environment for three column entries

\newenvironment{header}{
    \setlength{\topsep}{0pt}\par\kern\topsep\centering\linespread{1.5}
}{
    \par\kern\topsep
} % new environment for the header

\newcommand{\placelastupdatedtext}{% \placetextbox{<horizontal pos>}{<vertical pos>}{<stuff>}
  \AddToShipoutPictureFG*{% Add <stuff> to current page foreground
    \put(
        \LenToUnit{\paperwidth-2 cm-0 cm+0.05cm},
        \LenToUnit{\paperheight-1.0 cm}
    ){\vtop{{\null}\makebox[0pt][c]{
        \small\color{gray}\textit{Last updated in September 2024}\hspace{\widthof{Last updated in September 2024}}
    }}}%
  }%
}%

% save the original href command in a new command:
\let\hrefWithoutArrow\href

% new command for external links:


\begin{document}
    \newcommand{\AND}{\unskip
        \cleaders\copy\ANDbox\hskip\wd\ANDbox
        \ignorespaces
    }
    \newsavebox\ANDbox
    \sbox\ANDbox{$|$}

    \begin{header}
        \fontsize{25 pt}{25 pt}\selectfont Cristiano Filho

        \vspace{5 pt}

        \normalsize
        \mbox{Salvador, Bahia}%
        \kern 5.0 pt%
        \AND%
        \kern 5.0 pt%
        \mbox{\hrefWithoutArrow{mailto:cristianoliveira01.co@gmail.com}{cristianoliveira01.co@gmail.com}}%
        \kern 5.0 pt%
        \AND%
        \kern 5.0 pt%
        \mbox{\hrefWithoutArrow{tel:+55-71-98397-36-44}{+55 (71) 98397-3644}}%
        \kern 5.0 pt%
        \AND%
        \kern 5.0 pt%
        \mbox{\hrefWithoutArrow{https://cristianofilho.vercel.app/}{https://cristianofilho.vercel.app/}}%
        \kern 5.0 pt%
        \AND%
        \kern 5.0 pt%
        \mbox{\hrefWithoutArrow{https://github.com/CristianoFIlho}{https://www.linkedin.com/in/cristiano-filho/}}%
        \kern 5.0 pt%
        \AND%
        \kern 5.0 pt%
        \mbox{\hrefWithoutArrow{https://github.com/CristianoFIlho}{https://github.com/CristianoFIlho}}%
        \AND%
        \kern 5.0 pt%
        \mbox{\hrefWithoutArrow{https://www.salesforce.com/trailblazer/cristiano-filho}{https://www.salesforce.com/trailblazer/cristiano-filho}}%
    \end{header}

    \vspace{5 pt - 0.3 cm}


    \section{Resume}



        
        \begin{onecolentry}
             Como Consultor de Soluções Digitais II na Capgemini, trabalho com a Salesforce Sales Cloud, além de outras habilidades relevantes, para fornecer soluções digitais inovadoras para clientes B2B. Trabalho remotamente, contribuindo para projetos de migração e implementação relacionados ao ecossistema Salesforce e suas nuvens. Tenho experiência em SOQL, LWC e Apex Triggers, criando soluções personalizadas que atendam às necessidades específicas de cada cliente.
        \end{onecolentry}

        \vspace{0.2 cm}

        \begin{onecolentry}
            The boilerplate content was inspired by \href{https://github.com/dnl-blkv/mcdowell-cv}{Gayle McDowell}.
        \end{onecolentry}


    
    \section{Quick Guide}

    \begin{onecolentry}
        \begin{highlightsforbulletentries}


        \item Each section title is arbitrary and each section contains a list of entries.

        \item There are 7 unique entry types: \textit{BulletEntry}, \textit{TextEntry}, \textit{EducationEntry}, \textit{ExperienceEntry}, \textit{NormalEntry}, \textit{PublicationEntry}, and \textit{OneLineEntry}.

        \item Select a section title, pick an entry type, and start writing your section!

        \item \href{https://docs.rendercv.com/user_guide/}{Here}, you can find a comprehensive user guide for RenderCV.


        \end{highlightsforbulletentries}
    \end{onecolentry}

    \section{Education}



        
        \begin{twocolentry}{
            Jan 2017 – Dec 2024
        }
            \textbf{Universidade Católica de Salvador}, Bacharelado em Engenharia de Software\end{twocolentry}

        \vspace{0.10 cm}
        \begin{onecolentry}
            \begin{highlights}
                \item GPA: 7/10 (\href{https://inscricao.ucsal.br/wp-content/uploads/2023/09/MATRIZ-ENGENHARIA-DE-SOFTWARE-MATUTINO.pdf}{Matriz curricular})
                \item \textbf{Coursework:} Fundamentos de Programação, Engenharia de Requisitos, Metodologias de Desenvolvimento, Testes de Software, Inteligência Artificial e Aprendizado de Máquina, Gestão de Projetos de Software, Desenvolvimento de Aplicações Web e Móveis, Interação Humano-Computador (IHC), Manutenção de Software, Segurança de Software
            \end{highlights}
        \end{onecolentry}

        \vspace{0.10 cm}





    
    \section{Experience}



        
        \begin{twocolentry}{
            Jan 2022 – Atual
        }
            \textbf{Desenvolvedor Salesforce}, Capgemini SA -- Salvador, BR\end{twocolentry}

        \vspace{0.10 cm}
        \begin{onecolentry}
            \begin{highlights}
                \item Reduced time to render user buddy lists by 75\% by implementing a prediction algorithm
                \item Integrated iChat with Spotlight Search by creating a tool to extract metadata from saved chat transcripts and provide metadata to a system-wide search database
                \item Redesigned chat file format and implemented backward compatibility for search
            \end{highlights}
        \end{onecolentry}


    


    
    \section{Projects}



        
        \begin{twocolentry}{
            \href{https://www.irmaosgoncalves.com.br/}{irmaosgoncalves.com.br/}
        }
        
            \textbf{Irmãos Gonçalves e Jaú Serve}\end{twocolentry}

        \vspace{0.10 cm}
        \begin{onecolentry}
            \begin{highlights}
                \item B2B: Durante esse período, minha ênfase estava na nuvem Service Cloud da Salesforce.Também desenvolvi habilidades em Apex Programming, Visual Pages, Microsoft Visual Studio Code e outras áreas relevantes.
                \item Tools Used: Administrador Salesforce, Service Cloud, Flow. 
            \end{highlights}
        \end{onecolentry}
        


        \vspace{0.2 cm}

        \begin{twocolentry}
            Jun 2022 - Oct 2022
        \end{twocolentry}
        

        \begin{twocolentry}{
            \href{veste.com}{veste.com/}
        }
            \textbf{Veste}\end{twocolentry}

        \vspace{0.10 cm}
        \begin{onecolentry}
            \begin{highlights}
                \item Developed a desktop calendar with globally shared and synchronized calendars, allowing users to schedule meetings with other users
                \item Tools Used: Service Cloud
            \end{highlights}
        \end{onecolentry}


        \vspace{0.2 cm}

        \begin{twocolentry}{
            Jun 2023 - Jan 2024
        }
        
            \textbf{Bradesco Seguros}\end{twocolentry}

        \vspace{0.10 cm}
        \begin{onecolentry}
            \begin{highlights}
                \item Trabalhei com MuleSoft Anypoint Platform, Scrum e outras habilidades criações de APIs e integrações.Novamente, a colaboração remota foi essencial para o sucesso dos projetos entregando um case de sucesso ao cliente.
                \item Tools Used: MuleSoft, Anypoint Studio, DataWeave,Anypoint Exchange, API  Postman, Mule Runtime Engine, Jira
            \end{highlights}
        \end{onecolentry}


        \vspace{0.2 cm}

        \begin{twocolentry}{
            Fev 2024 - Jun 2024 
        }
        
            \textbf{Wolksvagen}\end{twocolentry}

        \vspace{0.10 cm}
        \begin{onecolentry}
            \begin{highlights}
                \item Built a UNIX-style OS with a scheduler, file system, text editor, and calculator
                \item Tools Used: Flow, SOQL, LWC, Apex
            \end{highlights}
        \end{onecolentry}        


        \vspace{0.2 cm}

        \begin{twocolentry}{
            Jun 2024 - atual
        }
        
            \textbf{Bradesco Banking}\end{twocolentry}

        \vspace{0.10 cm}
        \begin{onecolentry}
            \begin{highlights}
                \item Built a UNIX-style OS with a scheduler, file system, text editor, and calculator
                \item Tools Used: Apex Trigger, Procedures, Flow Triggers, OmniStudio, Seles Cloud, Git, Admnistrador Salasforce
            \end{highlights}
        \end{onecolentry}
        

        

    
    \section{Technologies}



        
        \begin{onecolentry}
            \textbf{Languages:} Apex, SOQL, SOSL, JavaScript
        \end{onecolentry}

        \vspace{0.2 cm}

        \begin{onecolentry}
            \textbf{Technologies:} LWC, FlowTrigger, Procedures, ApexTrigger e MuleSoft
        \end{onecolentry}

        \vspace{0.2 cm}   

               \begin{onecolentry}
            \textbf{Soft Skills:} APIs, Swagger, Postman Collections 
        \end{onecolentry}

        \vspace{0.2 cm}   
        
        \begin{onecolentry}
            \textbf{Cloud:} Service Cloud, Seles Cloud
        \end{onecolentry}


        \section{Certifications}



        
        \begin{onecolentry}
            \textbf{Salesforce:} Specialist AI, Data Cloud, App Builder, Developer I
        \end{onecolentry}

                \begin{onecolentry}
            \textbf{MuleSoft:} Mulesoft Certified Developer 
        \end{onecolentry}

    

    

\end{document}